% Options for packages loaded elsewhere
\PassOptionsToPackage{unicode}{hyperref}
\PassOptionsToPackage{hyphens}{url}
%
\documentclass[
]{article}
\usepackage{amsmath,amssymb}
\usepackage{iftex}
\ifPDFTeX
  \usepackage[T1]{fontenc}
  \usepackage[utf8]{inputenc}
  \usepackage{textcomp} % provide euro and other symbols
\else % if luatex or xetex
  \usepackage{unicode-math} % this also loads fontspec
  \defaultfontfeatures{Scale=MatchLowercase}
  \defaultfontfeatures[\rmfamily]{Ligatures=TeX,Scale=1}
\fi
\usepackage{lmodern}
\ifPDFTeX\else
  % xetex/luatex font selection
\fi
% Use upquote if available, for straight quotes in verbatim environments
\IfFileExists{upquote.sty}{\usepackage{upquote}}{}
\IfFileExists{microtype.sty}{% use microtype if available
  \usepackage[]{microtype}
  \UseMicrotypeSet[protrusion]{basicmath} % disable protrusion for tt fonts
}{}
\makeatletter
\@ifundefined{KOMAClassName}{% if non-KOMA class
  \IfFileExists{parskip.sty}{%
    \usepackage{parskip}
  }{% else
    \setlength{\parindent}{0pt}
    \setlength{\parskip}{6pt plus 2pt minus 1pt}}
}{% if KOMA class
  \KOMAoptions{parskip=half}}
\makeatother
\usepackage{xcolor}
\usepackage{color}
\usepackage{fancyvrb}
\newcommand{\VerbBar}{|}
\newcommand{\VERB}{\Verb[commandchars=\\\{\}]}
\DefineVerbatimEnvironment{Highlighting}{Verbatim}{commandchars=\\\{\}}
% Add ',fontsize=\small' for more characters per line
\newenvironment{Shaded}{}{}
\newcommand{\AlertTok}[1]{\textcolor[rgb]{1.00,0.00,0.00}{\textbf{#1}}}
\newcommand{\AnnotationTok}[1]{\textcolor[rgb]{0.38,0.63,0.69}{\textbf{\textit{#1}}}}
\newcommand{\AttributeTok}[1]{\textcolor[rgb]{0.49,0.56,0.16}{#1}}
\newcommand{\BaseNTok}[1]{\textcolor[rgb]{0.25,0.63,0.44}{#1}}
\newcommand{\BuiltInTok}[1]{\textcolor[rgb]{0.00,0.50,0.00}{#1}}
\newcommand{\CharTok}[1]{\textcolor[rgb]{0.25,0.44,0.63}{#1}}
\newcommand{\CommentTok}[1]{\textcolor[rgb]{0.38,0.63,0.69}{\textit{#1}}}
\newcommand{\CommentVarTok}[1]{\textcolor[rgb]{0.38,0.63,0.69}{\textbf{\textit{#1}}}}
\newcommand{\ConstantTok}[1]{\textcolor[rgb]{0.53,0.00,0.00}{#1}}
\newcommand{\ControlFlowTok}[1]{\textcolor[rgb]{0.00,0.44,0.13}{\textbf{#1}}}
\newcommand{\DataTypeTok}[1]{\textcolor[rgb]{0.56,0.13,0.00}{#1}}
\newcommand{\DecValTok}[1]{\textcolor[rgb]{0.25,0.63,0.44}{#1}}
\newcommand{\DocumentationTok}[1]{\textcolor[rgb]{0.73,0.13,0.13}{\textit{#1}}}
\newcommand{\ErrorTok}[1]{\textcolor[rgb]{1.00,0.00,0.00}{\textbf{#1}}}
\newcommand{\ExtensionTok}[1]{#1}
\newcommand{\FloatTok}[1]{\textcolor[rgb]{0.25,0.63,0.44}{#1}}
\newcommand{\FunctionTok}[1]{\textcolor[rgb]{0.02,0.16,0.49}{#1}}
\newcommand{\ImportTok}[1]{\textcolor[rgb]{0.00,0.50,0.00}{\textbf{#1}}}
\newcommand{\InformationTok}[1]{\textcolor[rgb]{0.38,0.63,0.69}{\textbf{\textit{#1}}}}
\newcommand{\KeywordTok}[1]{\textcolor[rgb]{0.00,0.44,0.13}{\textbf{#1}}}
\newcommand{\NormalTok}[1]{#1}
\newcommand{\OperatorTok}[1]{\textcolor[rgb]{0.40,0.40,0.40}{#1}}
\newcommand{\OtherTok}[1]{\textcolor[rgb]{0.00,0.44,0.13}{#1}}
\newcommand{\PreprocessorTok}[1]{\textcolor[rgb]{0.74,0.48,0.00}{#1}}
\newcommand{\RegionMarkerTok}[1]{#1}
\newcommand{\SpecialCharTok}[1]{\textcolor[rgb]{0.25,0.44,0.63}{#1}}
\newcommand{\SpecialStringTok}[1]{\textcolor[rgb]{0.73,0.40,0.53}{#1}}
\newcommand{\StringTok}[1]{\textcolor[rgb]{0.25,0.44,0.63}{#1}}
\newcommand{\VariableTok}[1]{\textcolor[rgb]{0.10,0.09,0.49}{#1}}
\newcommand{\VerbatimStringTok}[1]{\textcolor[rgb]{0.25,0.44,0.63}{#1}}
\newcommand{\WarningTok}[1]{\textcolor[rgb]{0.38,0.63,0.69}{\textbf{\textit{#1}}}}
\setlength{\emergencystretch}{3em} % prevent overfull lines
\providecommand{\tightlist}{%
  \setlength{\itemsep}{0pt}\setlength{\parskip}{0pt}}
\setcounter{secnumdepth}{-\maxdimen} % remove section numbering
\ifLuaTeX
  \usepackage{selnolig}  % disable illegal ligatures
\fi
\IfFileExists{bookmark.sty}{\usepackage{bookmark}}{\usepackage{hyperref}}
\IfFileExists{xurl.sty}{\usepackage{xurl}}{} % add URL line breaks if available
\urlstyle{same}
\hypersetup{
  hidelinks,
  pdfcreator={LaTeX via pandoc}}

\author{}
\date{}

\begin{document}

\textbf{2.1-1
以图2-2为模型,说明INSERTION-SORT在数组A=\textless31,41,59,26,41,58\textgreater 上的执行过程。}

简要说明如下:

31 41 59 26 41 58

26 31 41 59 41 58

26 31 41 41 59 58

26 31 41 41 58 59

\textbf{2.1-2
重写过程INSERTION-SORT,使之按非升序(而不是非降序)排序。}

将第五行中的A{[}i{]}\textgreater key改为A{[}i{]}\textless key。

\textbf{2.1-3 考虑以下查找问题:}

\textbf{输入:n个数的一个序列A=\textless{}\(a_1,a_2, ……,a_n\)\textgreater 和一个值v。}

\textbf{输出:下标i使得v=A{[}i{]}或者当v不在A中出现时,v为特殊值NIL。}

\textbf{写出线性查找的伪代码,它扫描整个序列来查找v。使用一个循环不变式来证明你的算法是正确的。确保你的循环不变式满足三条必要的性质。}

\begin{Shaded}
\begin{Highlighting}[]
\NormalTok{LINEAR{-}SEARCH(A, v)}
\NormalTok{    for i = 1 to A.length}
\NormalTok{        if A[i] == v }
\NormalTok{            return i}
\NormalTok{    return "NIL"}
\end{Highlighting}
\end{Shaded}

循环不变式:A{[}1\ldots\ldots i-1{]}。

1.初始化:在循环开始前,因为未经过比较,可以确保循环不变式中没有等于v的元素;

2.保持:当下标为i的元素和v相等时,算法结束,否则继续判断下一个元素,直到序列中没有元素为止。因此循环不变式中没有与v相等的元素。

3.终止:由上述两点可以推出算法是正确的。

\textbf{2.1-4
考虑把两个n位二进制整数加起来的问题,这两个整数分别存储在两个n元数组A和B中。这两个整数的和应按二进制形式存储在一个(n+1)元数组C中。请给出该数组的形式化描述,并写出伪代码。}

形式化描述:

输入:两个n元数组A=\textless{}\(a_1, a_2, ..., a_n\)\textgreater 和B=\textless{}\(b_1, b_2, ..., b_n\)\textgreater,其中\(a_i\)和\(b_i\)表示二进制数的第i位。

输出:一个(n+1)元数组C=\textless{}\(c_1, c_2, ..., c_{n+1}\)\textgreater,其中\(c_i\)表示二进制和的第i位。

\begin{Shaded}
\begin{Highlighting}[]
\NormalTok{BINARY{-}ADDITION(A, B)}
\NormalTok{    n = A.length}
\NormalTok{    carry = 0}
\NormalTok{    C = new array of size (n + 1)}
    
\NormalTok{    for i = n downto 1 }
\NormalTok{        sum = A[i] + B[i] + carry}
\NormalTok{        C[i+1] = sum \% 2}
        
\NormalTok{        if sum \textgreater{}= 2}
\NormalTok{            carry = 1}
\NormalTok{        else:}
\NormalTok{            carry = 0}
    
\NormalTok{    C[1] = carry  }
    
\NormalTok{    return C}
\end{Highlighting}
\end{Shaded}


\end{document}
